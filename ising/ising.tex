\documentclass[letterpaper,twocolumn,amsmath,amssymb,pre,aps,10pt]{revtex4-1}
\usepackage{graphicx} % Include figure files
\usepackage{color}
\usepackage{nicefrac} % Include for inline fractions

\usepackage{xargs}                      % Use more than one optional parameter in a new commands
\usepackage[pdftex,dvipsnames]{xcolor}  % Coloured text etc.
\usepackage[colorinlistoftodos,prependcaption,textsize=normalsize]{todonotes}
\usepackage{mdframed}
\usepackage{braket}

\newcommand{\red}[1]{{\bf \color{red} #1}}
\newcommand{\green}[1]{{\bf \color{green} #1}}
\newcommand{\blue}[1]{{\bf \color{blue} #1}}
\newcommand{\cyan}[1]{{\bf \color{cyan} #1}}

\newcommand{\davidsays}[1]{{\color{red} [\green{David:} \emph{#1}]}}
\newcommand{\jpsays}[1]{{\color{red} [\blue{Jordan:} \emph{#1}]}}

\begin{document}
\title{Flat-histogram method comparison on the 2D Ising model
}

\author{Jordan K. Pommerenck} \author{David Roundy}
\affiliation{Department of Physics, Oregon State University,
  Corvallis, OR 97331}

\begin{abstract}
We compare the convergence of several flat-histogram methods applied
to the 2D Ising model, including the recently introduced stochastic
approximation with a dynamic update factor (SAD) method.  We compare
this method with the Wang-Landau (WL) method, the $1/t$ variant of
the WL method, and standard stochastic approximation Monte Carlo
(SAMC).  In addition, we consider a procedure WL followed by a
``production run'' with fixed weights that refines the estimation of
the entropy.  To our knowledge, this work is the first to test this
approach against other methods. We find that WL followed by a production run \emph{does} converge to
the true density of states, in contrast to pure WL.  Three of the
methods converge robustly: SAD, $1/t$-WL, and WL followed by a
production run.  Of these, SAD does not require \emph{a priori}
knowledge of the energy range.
\end{abstract}

\maketitle

\input{ising-body.tex}


\bibliography{ising} % Produces the bibliography via BibTeX.

\end{document}
