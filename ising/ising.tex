\documentclass[letterpaper,twocolumn,amsmath,amssymb,pre,aps,10pt]{revtex4-1}
\usepackage{graphicx} % Include figure files
\usepackage{color}
\usepackage{nicefrac} % Include for inline fractions

\usepackage{xargs}                      % Use more than one optional parameter in a new commands
\usepackage[pdftex,dvipsnames]{xcolor}  % Coloured text etc.
\usepackage[colorinlistoftodos,prependcaption,textsize=normalsize]{todonotes}
\usepackage{mdframed}
\usepackage{braket}

% define colors for comments
\definecolor{dark-gray}{gray}{0.10}
\definecolor{light-gray}{gray}{0.70}

\newcommand{\red}[1]{{\bf \color{red} #1}}
\newcommand{\green}[1]{{\bf \color{green} #1}}
\newcommand{\blue}[1]{{\bf \color{blue} #1}}
\newcommand{\cyan}[1]{{\bf \color{cyan} #1}}

\newcommand{\davidsays}[1]{{\color{red} [\green{David:} \emph{#1}]}}
\newcommand{\jpsays}[1]{{\color{red} [\blue{Jordan:} \emph{#1}]}}
\newcommandx{\jpcom}[2][1=inline]{\todo[linecolor=gray,backgroundcolor=light-gray,bordercolor=dark-gray,#1]{\textbf{Jordan says:} #2} }
\begin{document}

\title{Flat-histogram method comparison on the 2D Ising model
}

\author{Jordan K. Pommerenck} \author{David Roundy}
\affiliation{Department of Physics, Oregon State University,
  Corvallis, OR 97331}

\begin{abstract}
We examine the convergence of stochastic approximation with a dynamic
update factor (SAD) with other flat-histogram methods applied to the 2D
Ising model. The comparison with stochastic approximation Monte Carlo (SAMC)
and Wang-Landau (WL) methods shows that SAD performs robustly and without user
input knowledge of an energy range. We also compare SAD with WL followed by a
``production run'' with fixed weights which is the first comparison of its kind
recorded. We find that WL followed by a production run is preferable to pure WL
for preserving ergodicity and detailed balance. Since SAD is more powerful in
the common case in which the range of energies is not known in advance, the 2D
Ising model presents a unique challenge since the energy range is readily known
and easily discovered by WL methods.
\end{abstract}

\maketitle

\input{ising-body.tex}

\bibliography{ising} % Produces the bibliography via BibTeX.

\end{document}
