In the first manuscript \emph{Stochastic approximation Monte Carlo with a
dynamic update factor}, we developed and refined a novel Monte Carlo variant of
SAMC called SAD. We compared the convergence properties of SAD to a variety of
other weight-based flat-histogram methods. We also examined the convergence
behavior applied to the square-well fluid and a Lennard-Jones cluster. We found
that the method SAD performs robustly on both systems independent of
scaling. SAD effectively and efficiently samples the entire energy space defined
by a chosen temperature range.

In addition, we found that SAD converges far more rapidly than SAMC. The
convergence of SAD rivals WL methods, which were provided the energy range of
interest \emph{a priori}. Unlike WL, which can converge to the wrong density of
states, SAD shares the proven convergence properties of SAMC. We found that,
while SAMC converges for a reasonable choice of $t_0$, this nonphysical
parameter can be especially difficult to tune. Even for the best choice of
$t_0$, SAMC does not converge as rapidly as SAD or the WL methods, when a
relatively small range of energies is required, because the algorithm simulates
all possible energies.

We also found that SAD is far more powerful in the common scenario where the
range of energies is not known in advance. For the 31 atom Lennard-Jones
cluster system, the WL and $1/t$-WL algorithms showed convergence properties
that were highly dependent on the choice of energy range examined. This was
particularly problematic when examining heat capacity at low temperatures,
where it was difficult to determine the lowest energy that would have a
significant impact at the temperatures of interest.

In the second manuscript \emph{Flat-histogram method comparison on 2D Ising
model}, we tested SAD on the 2D Ising model against a number of weight-based
methods including a ``production'' run WL. We also examined the convergence
behavior applied to the 2D Ising model. We found that SAD, $1/t$-WL, and WL
followed by a production run (with an adequately small $\gamma^{\min}$)
demonstrate excellent and robust convergence.

We found that for larger Ising systems, SAD reduced the update factor more
slowly (and conservatively) than $1/t$-WL and WL followed by a production run.
However, the WL methods solved a simpler problem because they were given the
energy range \emph{a priori}, rather than a temperature range of interest such
as SAD requires. The 2D Ising model is a system where the energy range is known
\emph{a priori.}, which made a comparison between WL methods and SAD more
biased in favor of WL methods. We also showed in this work that WL followed by
a production run performed extremely well and is preferable to pure WL for
ensuring both ergodicity and detailed balance.

In the third manuscript \emph{An upper bound to gas delivery via pressure-swing
adsorption in nanoporous materials}, we developed an upper bound on the
deliverable capacity via pressure-swing adsorption in rigid porous solids and
compared the theory with experimental properties of pure gases. Although the
upper bound does not rule out the discovery of materials that reach current DOE
targets for room temperature hydrogen and methane storage, it casts strong
doubt on the possibility of achieving these goals when considering the
additional constraints imposed due to steric hindrance between substrate atoms
and the adsorbate.

Additionally, we found that the upper bound helps by directing research toward
materials with an ideal energy of attraction as computed from the theory. Our
upper bound can also readily be applied to the storage of other gases of
interest (beyond hydrogen and methane), with the proviso that the gas is far
from crystallization.

A driving motivation for the development of the novel Monte Carlo method SAD
and a theoretical upper bound on the deliverable capacity in rigid porous
solids is to allow the simulation of key thermodynamic properties of porous
materials. By computationally determining a materials deliverable capacity,
researchers save countless development and experimental hours. In particular,
this body of research heavily contributes to enabling light-vehicle gas storage
applications. However, the are two important research extensions that remain in
order to completely develop a method suitable for simulating the thermodynamic
properties of interest for any porous material. (i) The first is the
development of SAD with a minimum pressure $p_{\min}$ instead of a $T_{\min}$.
By varying the number of particles, the simulation can yield thermodynamic
insights for all densities (which is of necessary for computing the deliverable
capacity). (ii) The second step that remains is the merger of each SAD method
such that a multidimensional 2D SAD is formed. Such a Monte Carlo method would
be an invaluable tool for helping to enable light-vehicle gas storage
technology.