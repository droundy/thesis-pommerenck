%\thispagestyle{plain}
\thispagestyle{empty}
\begin{center}
	\large
	AN ABSTRACT OF THE DISSERTATION OF
\end{center}


\justify{}
\underline{Jordan K. Pommerenck} for the degree of \underline{Doctor of Philosophy} in \underline{Physics}
presented on June 8, 2020
\vspace{1.0cm}

\justify{}
Title: \underline{Advancing renewable gas storage using flat-histogram methods}.
\vspace{2.0cm}

\justify{}
Abstract approved:
\underline{\hspace{11cm}}
\vspace{0.1cm}

%\centering{}
\hspace{7.0cm} David J. Roundy
\vspace{1.0cm}

\justify{}
\doublespacing
This work introduces the novel flat-histogram Monte
Carlo (MC) method stochastic approximation with a dynamic update factor (SAD) and explores the convergence properties of a
variety of related weight-based MC methods. The new method is applied
to a number of physical `test’ systems including the 2D Ising model, a
square-well fluid, and a Lennard-Jones cluster. A driving motivation for
developing novel Monte Carlo methods that have physically based tunable
parameters is to provide simulation methods for exploring metal-organic
frameworks (MOFs). A theoretical framework for gas adsorption and
delivery in metal-organic frameworks is developed and compared with
experimental data. SAD is introduced as a powerful method for examining
thermodynamic properties for MOFs. A preliminary groundwork is laid for
the future development of a multi-dimensional SAD which would be able
to compute all properties of interest for any given MOF thereby providing a powerful way for researchers to determine any porous media's suitability for gas storage applications.

\newpage{}
\thispagestyle{empty}
\singlespacing

\vspace*{4.0cm}
\begin{center}
$\copyright$ Copyright by Jordan K. Pommerenck \\
June 8, 2020 \\
All Rights Reserved
\end{center}

\newpage{}
\thispagestyle{empty}
\singlespacing

\begin{center}
Advancing renewable gas storage using flat-histogram methods

\vspace{1.0cm}
by \\
Jordan K. Pommerenck \\
\vspace{3.0cm}
A DISSERTATION \\
\vspace{0.5cm}
submitted to \\
\vspace{0.5cm}
Oregon State University \\
\vspace{3.0cm}
in partial fulfillment of \\
the requirements for the \\
degree of \\
\vspace{1.0cm}
Doctor of Philosophy \\
\vspace{3.0cm}
Presented June 8, 2020  \\
Commencement June 2020
\end{center}

\newpage{}
\thispagestyle{empty}
\singlespacing
\justify{}
\underline{Doctor of Philosophy} dissertation of \underline{Jordan K. Pommerenck} presented on
\underline{June 8, 2020.} \\

\justify{}
\vspace{0.5cm}
APPROVED: \\

\justify{}
\underline{\hspace{15cm}}
\begin{flushleft}
Major Professor, representing Physics
\vspace{1.0cm}
\end{flushleft}

\justify{}
\underline{\hspace{15cm}}
\begin{flushleft}
Chair of the Department of Physics
\vspace{1.0cm}
\end{flushleft}

\justify{}
\underline{\hspace{15cm}}
\begin{flushleft}
Dean of the Graduate School
\vspace{1.0cm}
\end{flushleft}

\vspace{3.0cm}
\justify{}
I understand that my dissertation will become part of the permanent collection
of Oregon State University libraries. My signature below authorizes release of my dissertation to any reader upon request. \\
\vspace{0.5cm}
\begin{flushleft}
\underline{\hspace{15cm}}
\end{flushleft}
\centering{}
Jordan K. Pommerenck, Author

\newpage{}
\thispagestyle{empty}
\begin{center}
	\large
	ACKNOWLEDGEMENTS
\end{center}
\justify{}
\doublespacing
I gratefully acknowledge my graduate advisor David Roundy for his
constant support and guidance throughout my research. His breadth of knowledge
and hands-off approach to advising have tremendously helped my development as an independent scientific researcher. I express thanks to my entire graduate committee: David McIntyre, Davide Lazzati, Yun-Shik Lee, and graduate council representative Chih-hung (Alex) Chang.
I would like to acknowledge the many outstanding undergraduate and graduate students that have positively impacted me personally and professionally while at Oregon State University. Finally, I would like to thank my parents for their unconditional support throughout my Ph.D. research.

\newpage{}
\thispagestyle{empty}
\begin{center}
	\large
	CONTRIBUTION OF AUTHORS
\end{center}
\justify{}
\doublespacing
Michael A. Perlin worked on much of the early C{}\verb!++! codebase. This codebase was used primarily in the early study of many of the MC methods before switching to RUST. In addition, he helped in the editing phase during the preparation of the first manuscript \emph{Stochastic approximation Monte Carlo with a dynamic update factor}. 
Tanner T. Simpson performed many of the early simulations for the flat-histogram methods using the C{}\verb!++! codebase. He also developed using Python some of the plotting scripts that allowed the saved C{}\verb!++! data to be visualized. In addition, he helped in the editing phase during the preparation of the first manuscript \emph{Stochastic approximation Monte Carlo with a dynamic update factor}.
Cory M. Simon collaborated together with us on the third manuscript \emph{An upper bound to gas delivery via pressure-swing adsorption in nanoporous materials}. He contributed significantly to the introduction section of the paper and helped in the editing phase throughout preparation. In addition, he designed the first figure in the paper and the first figure in the supplemental to the paper. 
