\documentclass[letterpaper,twocolumn,amsmath,amssymb,pre,aps,10pt]{revtex4-1}
\usepackage{graphicx} % Include figure files
\usepackage{color}
\usepackage{nicefrac} % Include for inline fractions

\usepackage{xargs}                      % Use more than one optional parameter in a new commands
\usepackage[pdftex,dvipsnames]{xcolor}  % Coloured text etc.
\usepackage[colorinlistoftodos,prependcaption,textsize=normalsize]{todonotes}
\usepackage{mdframed}

% define colors for comments
\definecolor{dark-gray}{gray}{0.10}
\definecolor{light-gray}{gray}{0.70}

\newcommand{\red}[1]{{\bf \color{red} #1}}
\newcommand{\green}[1]{{\bf \color{green} #1}}
\newcommand{\blue}[1]{{\bf \color{blue} #1}}
\newcommand{\cyan}[1]{{\bf \color{cyan} #1}}

\newcommand{\davidsays}[1]{{\color{blue} [\green{David:} \emph{#1}]}}
\newcommand{\jpsays}[1]{{\color{red} [\blue{Jordan:} \emph{#1}]}}
\newcommandx{\jpcom}[2][1=inline]{\todo[linecolor=gray,backgroundcolor=light-gray,bordercolor=dark-gray,#1]{\textbf{Jordan says:} #2} }
\begin{document}

\title{Stochastic Approximation Monte Carlo with a Dynamic Update
Factor
}

\author{Jordan K. Pommerenck} \author{Tanner T. Simpson}
\author{Michael A. Perlin} \author{David Roundy}
\affiliation{Department of Physics, Oregon State University,
  Corvallis, OR 97331}

\begin{abstract}
  We present a new Monte Carlo algorithm based on the Stochastic
  Approximation Monte Carlo (SAMC) algorithm for directly calculating
  the density of states. The proposed method is Stochastic
  Approximation with a Dynamic update factor (SAD)
  which dynamically adjusts the update factor $\gamma_t$ during the course of
  the simulation. We test this method on the square-well fluid and
  the 31-atom Lennard-Jones cluster and
  compare the convergence behavior of several
  related
  Monte Carlo methods. We find that both the SAD and $1/t$-Wang-Landau ($1/t$-WL)
  methods rapidly converge to the
  correct density of states without the need for the user to specify an
  arbitrary tunable parameter $t_0$ as in the case of SAMC.  SAD requires
  as input the temperature range of interest, in contrast to
  $1/t$-WL, which requires that the user identify the interesting range
  of energies.
  %
  The convergence of the $1/t$-WL method is very sensitive to the energy
  range chosen for the low-temperature heat capacity of the
  Lennard-Jones cluster.
  %
  Thus, SAD is more powerful in the common case in which the range
  of energies is not known in advance.
\end{abstract}

\maketitle

\input{sq-well-body.tex}

\bibliography{sq-well}% Produces the bibliography via BibTeX.

\end{document}
