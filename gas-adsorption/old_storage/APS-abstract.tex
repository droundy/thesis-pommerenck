\documentclass[letterpaper,onecolumn,amsmath,amssymb,jcp,aps,12pt]{revtex4-1}
\usepackage{xargs}      % Use more than one optional parameter in a new commands
\usepackage{titlesec}
\newcommand{\sectionbreak}{\clearpage}
\usepackage{color}
%\usepackage[utf8]{inputenc}
%\usepackage[english]{babel}
\usepackage[document]{ragged2e}

\begin{document}

\pagenumbering{gobble}
\begin{center}
    \textbf{Abstract}: Limited to 1300 characters
\end{center}

\center{}
\textbf{Authors:} Jordan K. Pommerenck, Cory M. Simon, David J. Roundy

\justify{}
The transportation sector accounts for 38\% of US energy-related carbon dioxide emissions \cite{useia} and generates toxic air pollution (particulate matter, ozone, NO$_x$, SO$_x$, carbon monoxide, volatile organic compounds) \cite{caiazzo2013air}. Alternative transportation fuels, such as natural gas or hydrogen, are therefore critical to mitigate climate change \cite{mcglade2015geographical} and improve air quality.
%
Both natural gas and hydrogen (gas) possess a very low volumetric energy density compared to (liquid) gasoline. Consequently, under storage space constraints in passenger vehicles, they must be densified such as through physical adsorption on nanoporous materials \cite{schoedel2016role} in order to drive a reasonable distance on a full tank of fuel.  
%
The US Department of Energy (DOE) set storage targets for adsorbed natural gas and hydrogen onboard light vehicles.
To assess the feasibility of these targets, we provide a theoretical upper bound on the density of natural gas and hydrogen that can be stored in and delivered by nanoporous materials via a pressure swing.
We conclude that, while the DOE storage targets are theoretically possible, the material would require a void fraction that is outside the range of void fractions in known materials exhibiting sufficient interactions with the gas.

\justify{}
\textbf{Keywords:} MOF, natural gas, hydrogen, void fraction, vehicular storage

% used if I wanted a new page with new section and numbering
%\section{NextSection}
%\pagenumbering{arabic}

\bibliography{bibfile}
\end{document}