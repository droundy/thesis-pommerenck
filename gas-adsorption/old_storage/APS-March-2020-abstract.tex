\documentclass[letterpaper,onecolumn,amsmath,amssymb,jcp,aps,12pt]{revtex4-1}
\usepackage{xargs}      % Use more than one optional parameter in a new commands
\usepackage{titlesec}
\newcommand{\sectionbreak}{\clearpage}
\usepackage{color}
%\usepackage[utf8]{inputenc}
%\usepackage[english]{babel}
\usepackage[document]{ragged2e}

\begin{document}

\pagenumbering{gobble}
\begin{center}
    \textbf{Abstract}: Limited to 1300 characters
\end{center}

\center{}
\textbf{Authors:} Jordan K. Pommerenck, Cory M. Simon, David J. Roundy

\justify{}
The transportation sector accounts for 38\% of US energy-related carbon dioxide emissions \cite{useia} and generates toxic air pollution (particulate matter, ozone, NO$_x$, SO$_x$, carbon monoxide, volatile organic compounds) \cite{caiazzo2013air}. Alternative transportation fuels, such as natural gas and hydrogen, are therefore critical to mitigate climate change \cite{mcglade2015geographical} and improve air quality.
%
Since they possess a low volumetric energy density compared to (liquid) gasoline, both natural gas and hydrogen gas must be densified such as through physical adsorption on nanoporous materials \cite{schoedel2016role} in order for reasonable light vehicle operating range.  
%
The US Department of Energy (DOE) has set on-board light vehicle storage targets for adsorbed natural gas and hydrogen. We provide a theoretical upper bound for the deliverable capacity of natural gas and hydrogen that applies to any uniform potential nanoporous material via pressure swing adsorption. We compare experimental data for MOFs to the upper bound with $q_\text{st}$ converted to $G_\text{st}$. We conclude that the DOE storage targets are theoretically impossible for room temperature hydrogen storage but could certainly be met by increasing the release temperature above the storage temperature. We also conclude that while the DOE storage targets are theoretically possible for natural gas, any applicable adsorbent would need a prohibitively high void fraction.

\justify{}
\textbf{Keywords:} MOF, natural gas, hydrogen, upper bound, vehicular storage

\bibliography{bibfile}
\end{document}