\documentclass[letterpaper,twocolumn,amsmath,amssymb,pre,aps,10pt]{revtex4-1}
\usepackage{graphicx} % Include figure files
\usepackage{color}
\usepackage{nicefrac} % Include for inline fractions

\usepackage{xargs}                      % Use more than one optional parameter in a new commands
\usepackage[pdftex,dvipsnames]{xcolor}  % Coloured text etc.
\usepackage[colorinlistoftodos,prependcaption,textsize=normalsize]{todonotes}
\usepackage{mdframed}

% define colors for comments
\definecolor{dark-gray}{gray}{0.10}
\definecolor{light-gray}{gray}{0.70}

\newcommand{\red}[1]{{\bf \color{red} #1}}
\newcommand{\green}[1]{{\bf \color{green} #1}}
\newcommand{\blue}[1]{{\bf \color{blue} #1}}
\newcommand{\cyan}[1]{{\bf \color{cyan} #1}}

\newcommand{\davidsays}[1]{{\color{blue} [\green{David:} \emph{#1}]}}
\newcommand{\jpsays}[1]{{\color{red} [\blue{Jordan:} \emph{#1}]}}
\newcommandx{\jpcom}[2][1=inline]{\todo[linecolor=gray,backgroundcolor=light-gray,bordercolor=dark-gray,#1]{\textbf{Jordan says:} #2} }
\begin{document}

\title{Ideal choice of a constraining sphere in Lennard-Jones cluster
simulations using flat-histogram methods
}

\author{Jordan K. Pommerenck} \author{David Roundy}
\affiliation{Department of Physics, Oregon State University,
  Corvallis, OR 97331}

\begin{abstract}
  We study the impact of the Lennard-Jones cluster `constraining' radius on 
  the thermodynamic properties such as heat capacity. We use SAD. We examine
  the 38 atom Lennard-Jones cluster.
\end{abstract}

\maketitle

\section{Introduction}
% (1) After examination, a major triumph would be to determine $R_c$ theoretically particularly
% for large constraining radii i.e. extract volume independent heat capacity.

% (3) 2D histogram could give information out of all clusters for a single
% simulation (will it converge?).

Lennard-Jones (LJ) clusters have been extensively studied using both Monte Carlo methods~\cite{frantsuzov2005size, mandelshtam2006structural, mandelshtam2006multiple} and molecular dynamics~\cite{honeycutt1987molecular, calvo1995configurational, calvo2000phase} as global optimization problems~\cite{wales1997global, wales1998global, wales1999global, doye1999double} and systems with finite size phase transitions~\cite{neirotti2000phase, sabo2005pressure, sehgal2014phase}. The potential energy of a cluster interacting with the LJ potential can
be written in terms of a well depth $\epsilon$ and pair separation $2^{1/6}\sigma$:
\begin{align}
E = 4\epsilon \sum_{i<j} \left[ \left( \frac{\sigma}{r_{ij}} \right)^{12} - \left( \frac{\sigma}{r_{ij}} \right)^{6} \right]
\end{align}

Over the last several decades, the global minima of various clusters with a number of atoms less than 1000 have been
found using simulation techniques such as basin-hopping~\cite{wales1997global, yoo2003possible} and genetic algorithms~\cite{gregurick1996global, daven1996structural, doye1998thermodynamics}. Other Monte Carlo methods such as flat-histogram algorithms~\cite{poulain2006performances, pommerenck2020stochastic} and parallel tempering~\cite{mandelshtam2006multiple, mandelshtam2006structural} have been used to provide insight to the thermodynamic properties of small clusters such as heat capacity and latent heat. The heat capacity curves for various clusters can then be used to predict solid-solid and liquid-solid transitions in the cluster. 

% The latent heat of these transitions was computed as the area under each peak in the heat capacity curve. It was found that the
% latent heat associated with the first peak is approximately 0.67lm, where lm = 0.29/Nǫ is the latent heat of the second peak,
% which, as we will show below, signals the melting of the cluster. (N = 309  https://arxiv.org/pdf/cond-mat/0512147.pdf)

% Basin hopping apparently uses a MC simulation with an R_c radius but only used 
A difficulty in Monte Carlo simulation of LJ clusters is the choice of a `constraining' radius $R_c$ necessary to prevent cluster dissociation.
The choice of an ideal $R_c$ presents critical problems in terms of accurately simulating thermodynamic
properties. Ideally, the user would like the radius to be as small as possible to decrease simulation time.
Unfortunately, too small of a radius for a given number N atoms results in incorrect simulation properties
such as shifted heat capacity peaks. Likewise, too large a radius results in longer simulation times
due to an increase in the time necessary to discover all energy states due to atom equilibration. The random
sampling of the energies can become increasingly difficult with the addition of the liquid-like and vapor regions~\cite{neirotti2000phase}.

Some research on LJ clusters choose a small constraining radius to achieve convergence in
a reasonable simulation time~\cite{neirotti2000phase}. Unfortunately, later research~\cite{frantz2001magic} confirmed that these radii can be too small
to ensure accurate results for any arbitrary cluster size but noted that they chose their own radii empirically. Additional work has been done to determine
an ideal $R_c$ for atomic clusters~\cite{yin2012massively} but for particle numbers much larger than $N=10$, the task of choosing an appropriate radius is complicated and empirical
in nature. 

The reason for this is that it is hard to choose a constraining sphere for any given LJ simulation that achieves ergodicity while not introducing fluctuations in the desired thermodynamic properties.

In this work, we explore the impact of the `constraining' sphere on heat capacity calculations using the flat-histogram
method SAD which we discuss in Sec.~\ref{sec:methods}.

\section{Methods}\label{sec:methods}
In this section, we introduce Stochastic Approximation Monte Carlo (SAMC) and
the refinement made by Pommerenck \emph{et al.}~\cite{pommerenck2020stochastic}
called Stochastic Approximation with a dynamic update factor (SAD). We start with a general description of SAMC and provide background and brief implementation scheme for SAD (for in-depth discussion the reader is encouraged to consult the original implementation~\cite{pommerenck2020stochastic}).

SAMC has a simple schedule by which the update factor $\gamma$ is continuously
decreased~\cite{liang2007stochastic, werlich2015stochastic,
schneider2017convergence}.  The update factor is defined in the
original implementation~\cite{liang2007stochastic} in terms of an
arbitrary tunable parameter $t_0$,
\begin{align}
\gamma_{t}^{\text{SA}} =\frac{t_0}{\max(t_0,t)}
\end{align}
where as above $t$ is the number of moves that have been attempted.

SAMC offers extreme simplicity and proven convergence.
Provided the update factor satisfies
\begin{align}
\sum_{t=1}^\infty \gamma_{t} = \infty \quad\textrm{and}\quad
\sum_{t=1}^\infty \gamma_{t}^\zeta < \infty
\end{align}
where $\zeta \in \{1,2\}$, Liang has shown that the weights are proven
to converge to the true density of states~\cite{liang2006theory,
liang2007stochastic}. The convergence time depends only on the choice of
parameter $t_0$.
The parameter $t_0$ can unfortunately be difficult to chose in advance
for arbitrary systems.
Liang \emph{et al.} give a rule of thumb in
which $t_0$ is chosen in the range from $2N_S$ to $100N_S$ where $N_S$
is the number of energy bins~\cite{liang2007stochastic}.  Schneider
\emph{et al.} found and later Pommerenck \emph{et al.} confirmed that for the Ising model this heuristic is helpful for small spin systems, but that larger systems require an even higher $t_0$ value~\cite{schneider2017convergence, pommerenck2020flat}.

Pommerenck \emph{et al.} proposed a refinement~\cite{pommerenck2020stochastic} to SAMC where the update factor is determined dynamically rather than by the user. Stochastic Approximation with a dynamic $\gamma$ (SAD) requires the user to provide the lowest temperature of interest $T_{\min}$ rather than the unphysical SAMC parameter $t_0$. The update factor
can be written in terms of the current estimates for the highest $E_H$ and
lowest $E_L$ energies of interest and the last time that an energy in the range
of interest is encountered $t_L$.
\begin{align}
  \gamma_{t}^{\text{SAD}} =
     \frac{
       \frac{E_{H}-E_{L}}{T_{\text{min}}} + \frac{t}{t_L}
     }{
       \frac{E_{H}-E_{L}}{T_{\text{min}}} + \frac{t}{N_S}\frac{t}{t_L}
     }
\end{align}
SAD only explores the energy range of interest as specified by the minimum
temperature of interest $T_{\min} < T < \infty$. During the simulation the two
energies $E_H$ and $E_L$, are refined such that the range of energies are conservatively
estimated. The weights are calculated for each energy region according to the original
prescription.
\begin{enumerate}
\item {$E < E_L$:} $w(E>E_H) = w(E_H)$
\item {$E_L < E < E_H$:} moves are handled the same as other weight-based
methods that are mentioned
\item {$E > E_H$:} $w(E<E_L) = w(E_L)e^{-\frac{E_L-E}{T_{\min}}}$
\end{enumerate}
Each time the simulation changes the value of $E_H$ or $E_L$, the weights
within the new portion of the interesting energy range are updated.

\section{Results}
\jpsays{THE RESULTS GO HERE}
We examine the impact of the `constraining' radius on the convergence of the heat capacity for a 38 atom LJ cluster.

\section{Conclusions}
\jpsays{THE CONCLUSTION GOES HERE}

\jpsays{(Need to fix lj-comparison.py to plot data probably manually is best by command line with argparse)
N = 38 for the radius comparison.}

\bibliography{lj-cluster}% Produces the bibliography via BibTeX.

\end{document}
